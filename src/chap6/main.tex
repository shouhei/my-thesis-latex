\chapter{考察}

本章では、前章での評価実験での結果を受け提案する授業の継続すべき点、問題点、改良点について述べる。

\section{問題点}

とりあえず箇条書き

\begin{itemize}
  \item[・] 技術選定の余地があったのではないか?
  \item[・] 1コマとしては取り扱い分量が多い
    \begin{itemize}
    \item[・] 以下本来取り扱うべきmin
      \begin{itemize}
        \item[・] WAFの使いかた
        \item[・] WAFを利用した際のテスト
        \item[・] 継続的インテグレーションの利用
      \end{itemize}
    \item[・] 別授業にするべき
      \begin{itemize}
        \item[・] プログラミング言語の扱い方
        \item[・] コマンドラインの利用方法
        \item[・] バージョン管理の方法
      \end{itemize}
    \end{itemize}
  \item[・] 受講人数が増えたら対応できない
    \begin{itemize}
      \item[・] TAの導入
      \item[・] 産学連携を視野に入れる
    \end{itemize}
\end{itemize}

\section{改良可能な点}

とりあえず箇条書き

\begin{itemize}
\item[・] 利用する*aaSの選定
  \begin{itemize}
    \item[・] GitHub Education (プライベートリポジトリが使えた)
    \item[・] GitHub Student Developer Pack (Travice CIがプライベートリポジトリで使えた)
  \end{itemize}
\item[・] 改善するソフトウェアのお題
\end{itemize}
