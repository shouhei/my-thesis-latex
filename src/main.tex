%%%%%%%%%%%%%%%%%%%%%%%%%%%%%%%%%%%%%%%
% プリアンブル 各種設定情報
%%%%%%%%%%%%%%%%%%%%%%%%%%%%%%%%%%%%%%%
\documentclass[a4paper,11pt,oneside,openany]{jsbook}
%
\usepackage{fancyhdr}
\usepackage{amsmath,amssymb}
\usepackage{bm}
\usepackage[dvipdfmx]{graphicx}
\usepackage{subfigure}
\usepackage{url}
\usepackage{verbatim}
\usepackage{wrapfig}
\usepackage{ascmac}
\usepackage{fancyvrb}
\usepackage{makeidx}
\usepackage{comment}
\usepackage{../style/eclbkbox}
\usepackage{float}

%%%%%%%%%%ここより上は必要なスタイルを読み込みたい時にのみ編集%%%%%%%%%%%%%
\usepackage{../style/myjlab}
\makeindex
%%%%%%%%%% 念のため \usepackegeの一番下に myjlab があるように%%%%%%%%%%%%%

%%%%%%%%%%基本情報設定変更の必要なし%%%%%%%%%%%%%%%%%%%%%%%%%%%%
\daigaku{青山学院大学}
\gakubu{社会情報学部}
\gakka{社会情報学科}
\syubetsu{卒業論文}
\labname{宮治研究室}
\chiefexaminer{宮治 裕 准教授}

%%%%%%%%%%%%%%%%%%%%%%%%%%%%%%%%%%%%%%%
% ここから先「ここまで個人設定」の範囲に
% 各自の固有の情報を記入して下さい
%%%%%%%%%%%%%%%%%%%%%%%%%%%%%%%%%%%%%%%
\nendo{2015年度}
\teisyutsu{2016年~~1月}
\snum{38114004}
\jname{山口 将平}
\thesistitle{インターネットサービスを主軸としたIT企業の業務に基づく実践的ソフトウェア開発手法} %タイトルを記入
%\thesissubtitle{\LaTeX の利用} %サブタイトルを記入 ない場合はコメントアウト
%\SUBTtrue %サブタイトル有りの場合 ない場合は,コメントアウト
%\SUBTfalse %サブタイトル無しの場合 有る場合は,コメントアウト
%%%%%%%%%% ここまで個人設定 %%%%%%%%%%%%%%

%%%%%%%%%%%%%%%%%%%%%%%%%%%%%%%%%%%%%%%
% ここから先,論文内原稿
% 「ここまで共通」まで編集不要
%%%%%%%%%%%%%%%%%%%%%%%%%%%%%%%%%%%%%%%
\begin{document}
\linesparpage{30} %行数指定
\mojiparline{35} %文字数指定
\pagestyle{empty}
\maketitle

\frontmatter
%%% 論文要旨
\chapter*{論文要旨}
%\thispagestyle{empty}
\addcontentsline{toc}{chapter}{論文要旨}
\input{abstract}
\pagestyle{plain}
\pagenumbering{roman}
% abstract.texの中は \chapterなど書かずに単なるテキストを入力する

%%% 謝辞
\chapter*{謝辞}
%\thispagestyle{empty}
\addcontentsline{toc}{chapter}{謝辞}
\input{thanks}
% thanks.texの中は \chapterなど書かずに単なるテキストを入力する

%%% 目次
\tableofcontents
% 目次は自動生成される
%
\mainmatter
\pagestyle{fancy}
\pagenumbering{arabic}
%%%%%%%%%% 「ここまで共通」 %%%%%%%%%%%%%%


%%%%%%%%%%%%%%%%%%%%%%%%%%%%%%%%%%%%%%%
% ここから先「ここまで論文本文」の範囲を
% 各自の章立てや付録にあわせて編集して下さい
%%%%%%%%%%%%%%%%%%%%%%%%%%%%%%%%%%%%%%%

%%% 本文ここから chap1 chap2 chap3 同様に必要なだけ章を入れる
\include{./chap1} % 1章
\include{./chap2} % 2章
%%%%%%%%%%%%%%%%%%%%%%%%%%%%%%%%%%%%%%%
% プリアンブル 各種設定情報
%%%%%%%%%%%%%%%%%%%%%%%%%%%%%%%%%%%%%%%
\documentclass[a4paper,11pt,oneside,openany]{jsbook}
%
\usepackage{fancyhdr}
\usepackage{amsmath,amssymb}
\usepackage{bm}
\usepackage[dvipdfmx]{graphicx}
\usepackage{subfigure}
\usepackage{url}
\usepackage{verbatim}
\usepackage{wrapfig}
\usepackage{ascmac}
\usepackage{fancyvrb}
\usepackage{makeidx}
\usepackage{comment}
\usepackage{eclbkbox}
\usepackage{float}

%%%%%%%%%%ここより上は必要なスタイルを読み込みたい時にのみ編集%%%%%%%%%%%%%
\usepackage{../style/myjlab}
\makeindex
%%%%%%%%%% 念のため \usepackegeの一番下に myjlab があるように%%%%%%%%%%%%%

%%%%%%%%%%基本情報設定変更の必要なし%%%%%%%%%%%%%%%%%%%%%%%%%%%%
\daigaku{青山学院大学}
\gakubu{社会情報学部}
\gakka{社会情報学科}
\syubetsu{卒業論文}
\labname{宮治研究室}
\chiefexaminer{宮治 裕 准教授}

%%%%%%%%%%%%%%%%%%%%%%%%%%%%%%%%%%%%%%%
% ここから先「ここまで個人設定」の範囲に
% 各自の固有の情報を記入して下さい
%%%%%%%%%%%%%%%%%%%%%%%%%%%%%%%%%%%%%%%
\nendo{2013年度}
\teisyutsu{2014年~~1月}
\snum{15387019}
\jname{宮治 裕}
\thesistitle{宮治研における論文作成について} %タイトルを記入
\thesissubtitle{\LaTeX の利用} %サブタイトルを記入 ない場合はコメントアウト
\SUBTtrue %サブタイトル有りの場合 ない場合は,コメントアウト
%\SUBTfalse %サブタイトル無しの場合 有る場合は,コメントアウト
%%%%%%%%%% ここまで個人設定 %%%%%%%%%%%%%%

%%%%%%%%%%%%%%%%%%%%%%%%%%%%%%%%%%%%%%%
% ここから先,論文内原稿
% 「ここまで共通」まで編集不要
%%%%%%%%%%%%%%%%%%%%%%%%%%%%%%%%%%%%%%%
\begin{document}
\linesparpage{30} %行数指定
\mojiparline{35} %文字数指定
\pagestyle{empty}
\maketitle

\frontmatter
%%% 論文要旨
\chapter*{論文要旨}
%\thispagestyle{empty}
\addcontentsline{toc}{chapter}{論文要旨}
\input{abstract}
\pagestyle{plain}
\pagenumbering{roman}
% abstract.texの中は \chapterなど書かずに単なるテキストを入力する

%%% 謝辞
\chapter*{謝辞}
%\thispagestyle{empty}
\addcontentsline{toc}{chapter}{謝辞}
\input{thanks}
% thanks.texの中は \chapterなど書かずに単なるテキストを入力する

%%% 目次
\tableofcontents
% 目次は自動生成される
%
\mainmatter
\pagestyle{fancy}
\pagenumbering{arabic}
%%%%%%%%%% 「ここまで共通」 %%%%%%%%%%%%%%


%%%%%%%%%%%%%%%%%%%%%%%%%%%%%%%%%%%%%%%
% ここから先「ここまで論文本文」の範囲を
% 各自の章立てや付録にあわせて編集して下さい
%%%%%%%%%%%%%%%%%%%%%%%%%%%%%%%%%%%%%%%

%%% 本文ここから chap1 chap2 chap3 同様に必要なだけ章を入れる
\include{chap1} % 1章
\include{chap2} % 2章
\include{chap3} % 3章
%\include{chap4} % 4章
%\include{chap5} % 5章
%\include{chap6} % 6章

%%% 付録 -- 必要なければ以下を2行コメントアウト
\appendix
\include{appendixA}
%\include{appendixB} %必要に応じて付録の数を増やす

%\clearpage
%%%%%%%%%% ここまで論文本文 %%%%%%%%%%%%%%


% ************** ここから先の範囲は編集不要 ****************
%%% 参考文献
\bibliographystyle{junsrt}
\bibliography{myrefs}
% myrefs.bib の中はサンプルファイルを参考に記述

\newpage
\printindex
%
\end{document}
 % 3章
%%%%%%%%%%%%%%%%%%%%%%%%%%%%%%%%%%%%%%%
% プリアンブル 各種設定情報
%%%%%%%%%%%%%%%%%%%%%%%%%%%%%%%%%%%%%%%
\documentclass[a4paper,11pt,oneside,openany]{jsbook}
%
\usepackage{fancyhdr}
\usepackage{amsmath,amssymb}
\usepackage{bm}
\usepackage[dvipdfmx]{graphicx}
\usepackage{subfigure}
\usepackage{url}
\usepackage{verbatim}
\usepackage{wrapfig}
\usepackage{ascmac}
\usepackage{fancyvrb}
\usepackage{makeidx}
\usepackage{comment}
\usepackage{eclbkbox}
\usepackage{float}

%%%%%%%%%%ここより上は必要なスタイルを読み込みたい時にのみ編集%%%%%%%%%%%%%
\usepackage{../style/myjlab}
\makeindex
%%%%%%%%%% 念のため \usepackegeの一番下に myjlab があるように%%%%%%%%%%%%%

%%%%%%%%%%基本情報設定変更の必要なし%%%%%%%%%%%%%%%%%%%%%%%%%%%%
\daigaku{青山学院大学}
\gakubu{社会情報学部}
\gakka{社会情報学科}
\syubetsu{卒業論文}
\labname{宮治研究室}
\chiefexaminer{宮治 裕 准教授}

%%%%%%%%%%%%%%%%%%%%%%%%%%%%%%%%%%%%%%%
% ここから先「ここまで個人設定」の範囲に
% 各自の固有の情報を記入して下さい
%%%%%%%%%%%%%%%%%%%%%%%%%%%%%%%%%%%%%%%
\nendo{2013年度}
\teisyutsu{2014年~~1月}
\snum{15387019}
\jname{宮治 裕}
\thesistitle{宮治研における論文作成について} %タイトルを記入
\thesissubtitle{\LaTeX の利用} %サブタイトルを記入 ない場合はコメントアウト
\SUBTtrue %サブタイトル有りの場合 ない場合は,コメントアウト
%\SUBTfalse %サブタイトル無しの場合 有る場合は,コメントアウト
%%%%%%%%%% ここまで個人設定 %%%%%%%%%%%%%%

%%%%%%%%%%%%%%%%%%%%%%%%%%%%%%%%%%%%%%%
% ここから先,論文内原稿
% 「ここまで共通」まで編集不要
%%%%%%%%%%%%%%%%%%%%%%%%%%%%%%%%%%%%%%%
\begin{document}
\linesparpage{30} %行数指定
\mojiparline{35} %文字数指定
\pagestyle{empty}
\maketitle

\frontmatter
%%% 論文要旨
\chapter*{論文要旨}
%\thispagestyle{empty}
\addcontentsline{toc}{chapter}{論文要旨}
\input{abstract}
\pagestyle{plain}
\pagenumbering{roman}
% abstract.texの中は \chapterなど書かずに単なるテキストを入力する

%%% 謝辞
\chapter*{謝辞}
%\thispagestyle{empty}
\addcontentsline{toc}{chapter}{謝辞}
\input{thanks}
% thanks.texの中は \chapterなど書かずに単なるテキストを入力する

%%% 目次
\tableofcontents
% 目次は自動生成される
%
\mainmatter
\pagestyle{fancy}
\pagenumbering{arabic}
%%%%%%%%%% 「ここまで共通」 %%%%%%%%%%%%%%


%%%%%%%%%%%%%%%%%%%%%%%%%%%%%%%%%%%%%%%
% ここから先「ここまで論文本文」の範囲を
% 各自の章立てや付録にあわせて編集して下さい
%%%%%%%%%%%%%%%%%%%%%%%%%%%%%%%%%%%%%%%

%%% 本文ここから chap1 chap2 chap3 同様に必要なだけ章を入れる
\include{chap1} % 1章
\include{chap2} % 2章
\include{chap3} % 3章
%\include{chap4} % 4章
%\include{chap5} % 5章
%\include{chap6} % 6章

%%% 付録 -- 必要なければ以下を2行コメントアウト
\appendix
\include{appendixA}
%\include{appendixB} %必要に応じて付録の数を増やす

%\clearpage
%%%%%%%%%% ここまで論文本文 %%%%%%%%%%%%%%


% ************** ここから先の範囲は編集不要 ****************
%%% 参考文献
\bibliographystyle{junsrt}
\bibliography{myrefs}
% myrefs.bib の中はサンプルファイルを参考に記述

\newpage
\printindex
%
\end{document}
 % 4章
%\include{chap5} % 5章
%\include{chap6} % 6章

%%% 付録 -- 必要なければ以下を2行コメントアウト
\appendix
%..................................................................

%\include{appendixB} %必要に応じて付録の数を増やす

%\clearpage
%%%%%%%%%% ここまで論文本文 %%%%%%%%%%%%%%


% ************** ここから先の範囲は編集不要 ****************
%%% 参考文献
\bibliographystyle{junsrt}
\bibliography{./myrefs}
% myrefs.bib の中はサンプルファイルを参考に記述

\newpage
\printindex
%
\end{document}
