\section{企業におけるアジャイルソフトウェア開発の実践}

本節では企業がどのようなツールを用いてアジャイルを展開しているかの具体例について述べる.

昨今の企業のインターネットサービス開発では,全体像としては企業やプロジェクトによってことなるが以下\ref{fig:dev_env}のような環境を用意する.開発用のコンピューターから,プログラムの変更をコードホスティングサーバー\footnote{コードホスティングサーバーとは,ソースコードを中央集権的に管理するためのサーバーである.}に送信すると,そのイベントを受け取った継続的インテグレーションツールが自動的に必要な手続きを踏み,検証環境にプログラムを展開する.検証環境で実際にプログラムに問題が無いことを確認した後,コードホスティングサーバーから本番環境への適用が行われる.

\begin{figure}[H]
\centering
\includegraphics[height=8cm]{./assets/images/dev_env.png}
\caption{ソフトウェア開発の環境概要図}
\label{fig:dev_env}
\end{figure}

またそれらの環境構築する際に選定する必要のあるソフトウェアを列挙すると以下のようなものが挙げられる.

\begin{itemize}
 \item[・]プログラミング言語
   \begin{itemize}
    \item[・]アプリケーションフレームワーク
    \item[・]テスティングフレームワーク
    \item[・]アプリケーションサーバー \footnote{
      アプリケーションサーバーとは,Webサーバーと自身が書いたプログラムをブリッジするためのサーバーのこと.プログラミング言語Javaで利用できるアプリケーションサーバーであればTomcatが有名.
    }
   \end{itemize}
 \item[・]バージョン管理システム
   \begin{itemize}
     \item[・]ソースコードホスティングシステム
   \end{itemize}
 \item[・]継続的インテグレーションシステム
 \item[・]webサーバー
 \item[・]RDBMS
\end{itemize}

上記を図に表すと以下のようになり,破線の中が強い繋がり,矢印での繋がりが緩やかな繋がりとなる.

\begin{figure}[H]
  \begin{center}
    \includegraphics[height=8cm]{./assets/images/tech_relation.png}
    \caption{選定すべきソフトウェアの関係性}
    \label{fig:tech_relation}
  \end{center}
\end{figure}

図\ref{fig:tech_relation}を見れば分かる通り,プログラミング言語とアプリケーションフレームワーク,バージョン管理システムを選定すると,おおまかにどのようなものを選択したら良いかが決定される.どのプログラミング言語を選定するかは,既に言語を習得している人数や言語の習得難易度,言語を中心としたコミュニティやエコシステム\footnote{ライブラリの公開方法,インストールやアップデート,ライブラリ同士の依存関係の解決の仕組みなど,一つのソフトウェアを作る際に必要となる仕組みのつながり}が整備されているかに気を配らなければならない.またインターネットサービスを開発することに主眼を置くと,そのためのライブラリや共通規格が整備されているかも考慮する必要がある.

その中でも,本論文執筆の時点で,考慮すべき点をクリアしており,企業での採用事例として出てくることが多く、本研究でも採用するプログラミング言語のRubyの例をここでで紹介し,それ以外の代表例はAppendixに記載する.

\begin{table}[ht]
  \begin{center}
    \begin{tabular}{|c|c|}
      \hline
      分類 & ツール名 \\
      \hline
      プログラミング言語 & Ruby \\
      \hline
      アプリケーションフレームワーク& Ruby On Rails\\
      \hline
      テスティングフレームワーク & Rspec \\
      \hline
      アプリケーションサーバー & Unicorn \\
      \hline
      バージョン管理システム & Git \\
      \hline
      ソースコードホスティングサーバー & GitHub \\
      \hline
      継続的インテグレーションシステム & Jenkins \\
      \hline
      webサーバー & Nginx \\
      \hline
      RDBMS & MySQL \\
      \hline
    \end{tabular}
  \end{center}
\end{table}
