\section{提案を元にした評価実験に向けた授業シラバス}

本節では前節での提案を元に,評価実験を行う際の授業シラバスを提示する.

\subsection{時間}

本研究で提案する授業は,大学設置基準\cite{univ_criteria}で制定されている単位,2単位分の授業とする.

%\subsection{授業概要} 提案で既に記述済み
%
%本授業ではWebサービスを主事業としたIT企業で用いるソフトウェア開発の手法を手本とし、ソフトウェア開発に必要な知識や技術を身につける授業です。

\subsection{評価実験授業での技術選定}

企業で採用されている技術を元に、本授業で採用する技術的なツールは表\ref{tab:技術選定}のとおりとなる.

\begin{table}[ht]
  \begin{center}
    \caption{技術選定}
    \begin{tabular}{|c|c|c|}
      \hline
      分類 & ツール名 & 備考 \\
      \hline
      プログラミング言語 & Ruby & \\
      \hline
      アプリケーションフレームワーク & Ruby on Rails & \\
      \hline
      テスティングフレームワーク & Rspec & \\
      \hline
      アプリケーションサーバー &  & \\
      \cline{1-1}\cline{3-3}
      Webサーバー & Heroku & \\
      \hline
      RDBMS & PostgreSQL & \\
      \hline
      バージョン管理システム & Git & \\
      \hline
      ソースコードホスティングサーバー & GitHub & \\
      \hline
      継続的インテグレーションシステム & CircleCI & \\
      \hline
    \end{tabular}
    \label{tab:技術選定}
  \end{center}
\end{table}

\subsection{履修条件}

本研究の評価実験に向けた授業では対象\ref{target}で述べた条件に加え以下を履修条件とする。

\begin{itemize}
 \item[・] HTMLやCSSの扱い方を知っていること
 \item[・] HTTPメソッドやセッション,クッキーを理解していること
\end{itemize}

\subsection{授業計画}

本研究で提案する授業の計画は表\ref{tab:授業計画}のとおりとする.

\begin{table}[H]
  \begin{center}
    \caption{シラバス}
    \begin{tabular}{|c|c|c|c|}
      \hline
      授業回数 & 内容 & \ref{tech-content}取り扱い内容との対応 \\%& 備考 \\
      \hline
      第1回 & ソフトウェア開発について & 1 \\%&  \\
      \hline
      第2回 & コマンドラインの利用方法 &  \\%&  開発環境の説明も含めた復習 \\
      \hline
      第3回 & Ruby基礎文法 &  2 \\%&  \\
      \hline
      第4回 & オブジェクト指向Ruby & 2 \\%& Rubyの説明も兼ねたオブジェクト指向の復習 \\
      \hline
      第5回 & Gitの使いかた & 3 \\%& \\
      \hline
      第6回 & GitHubとRspec & 2,3,4 \\%&  \\
      \hline
      第7回 & Ruby On Rails の始め方 & 5 \\%&  \\
      \hline
      第8回 & Ruby On Rails を利用したソフトウェアの作成 & 5 \\%& \\
      \hline
      第9回 & Ruby On Rails を利用したソフトウェアの改善 1 & 5 \\%& \\
      \hline
      第10回 & Ruby On Rails を利用したソフトウェアの改善 2 & 5 \\%& \\
      \hline
      第11回 & Ruby On Rails とテスト,継続的インテグレーション & 4,5,6 \\%& \\
      \hline
      第12回 & Ruby On Rails とテスト,継続的インテグレーション & 4,5,6 \\%& \\
      \hline
      第13回 & Ruby On Rails で作成されたソフトウェアの改善実習 & 全内容 \\%& 画像投稿,お気に入りシステムの改善\\
      \hline
      第14回 & Ruby On Rails で作成されたソフトウェアの改善実習 & 全内容 \\%& 画像投稿,お気に入りシステムの改善\\
      \hline
      第15回 & Ruby On Rails で作成されたソフトウェアの改善実習 & 全内容 \\%& 画像投稿,お気に入りシステムの改善\\
      \hline
    \end{tabular}
    \label{tab:授業計画}
  \end{center}
\end{table}
