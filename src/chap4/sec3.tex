\section{実践と制約を踏まえたソフトウェア開発実習の提案}

前節までで企業で実践しているソフトウェア開発手法についてと,その開発手法を学校で取り扱う際の問題点について述べた.
本説では前節までのことを踏まえ,授業の目的と目標,前提条件をまとめた上で取り扱い内容とシラバスと授業の評価方法を提示する.

\subsection{授業の目的}

本研究で提案する授業の目的は,アジャイルソフトウェア開発に参加するための知識と技術を身につけることである.

\subsection{授業の目標}

本研究で提案する授業は,アジャイルソフトウェア開発とはどのようなものかを知り, Ruby on Railsを用いてソフトウェア開発が出来るようになることを目標とする.

\subsection{取り扱い内容}

本研究で提案する授業で取り扱う内容は以下のとおりである.

\begin{enumerate}
  \item アジャイルソフトウェア開発について
  \item プログラミング言語 Ruby
  \item Gitを用いたソフトウェアのバージョン管理
  \item Rspecを用いたテスト
  \item Ruby On Rails を用いたソフトウェア開発
  \item 継続的インテグレーションシステムを利用した継続的改善
\end{enumerate}

また上記を扱う上で選定する技術的なツールは以下のとおりとなる.

\begin{table}[ht]
  \begin{center}
    \begin{tabular}{|c|c|c|}
      \hline
      分類 & ツール名 & 備考 \\
      \hline
      プログラミング言語 & Ruby & \\
      \hline
      アプリケーションフレームワーク & Ruby On Rails & \\
      \hline
      テスティングフレームワーク & Rspec & \\
      \hline
      アプリケーションサーバー &  & \\
      \cline{1-1}\cline{3-3}
      Webサーバー & Heroku & \\
      \hline
      RDBMS & PostgreSQL & \\
      \hline
      バージョン管理システム & Git & \\
      \hline
      ソースコードホスティングサーバー & GitHub & \\
      \hline
      継続的インテグレーションシステム & CircleCI & \\
      \hline
    \end{tabular}
  \end{center}
\end{table}

\subsection{前提条件}

本研究で提案する授業を履修するにあたっての前提条件は以下のとおりとする.

\begin{itemize}
  \item[・] HTMLやCSSの扱い方を知っていること
  \item[・] HTTPメソッドやセッション,クッキーを理解していること
  \item[・] 任意のプログラミング言語を用いて,オブジェクト指向のためのプログラミング方法を知っていること
  \item[・] コマンドラインからLinuxを扱うことに抵抗が無いこと
\end{itemize}

\subsection{シラバス}

本研究で提案する授業のシラバスは以下のとおりとする.

\begin{table}[ht]
  \begin{center}
    \begin{tabular}{|c|c|c|c|}
      \hline
      授業回数 & 内容 & 取り扱い内容との対応 & 備考 \\
      \hline
      第1回 & ソフトウェア開発について & 1 &  \\
      \hline
      第2回 & コマンドラインの利用方法 &  &  開発環境の説明も含めた復習 \\
      \hline
      第3回 & Ruby基礎文法 &  2 &  \\
      \hline
      第4回 & オブジェクト指向Ruby & 2 & Rubyの説明も兼ねたオブジェクト指向の復習 \\
      \hline
      第5回 & Gitの使いかた & 3 & \\
      \hline
      第6回 & GitHubとRspec & 2,3,4 &  \\
      \hline
      第7回 & Ruby On Rails の始め方 & 5 &  \\
      \hline
      第8回 & Ruby On Rails を利用したソフトウェアの作成 & 5 & \\
      \hline
      第9回 & Ruby On Rails を利用したソフトウェアの改善 1 & 5 & \\
      \hline
      第10回 & Ruby On Rails を利用したソフトウェアの改善 2 & 5 & \\
      \hline
      第11回 & Ruby On Rails とテスト,継続的インテグレーション & 4,5,6 & \\
      \hline
      第12回 & Ruby On Rails とテスト,継続的インテグレーション & 4,5,6 & \\
      \hline
      第13回 & Ruby On Rails で作成されたソフトウェアの改善実習 & 全内容 & 画像投稿,お気に入りシステムの改善\\
      \hline
      第14回 & Ruby On Rails で作成されたソフトウェアの改善実習 & 全内容 & 画像投稿,お気に入りシステムの改善\\
      \hline
      第15回 & Ruby On Rails で作成されたソフトウェアの改善実習 & 全内容 & 画像投稿,お気に入りシステムの改善\\
      \hline
    \end{tabular}
  \end{center}
\end{table}

\subsection{評価方法}

本研究で提案する授業効果の評価は,終盤の3回分の実習で行うソフトウェア改善に対して環境 (図)を用意し収集できる以下の作業ログで行う.

\begin{table}[ht]
  \begin{center}
    \begin{tabular}{|c|c|}
      \hline
      収集できる物 & 評価方法 \\
      \hline
      Gitを利用したソースコードの変更ログ& \\
      \hline
      zsh\_historyを利用したコマンドの利用ログ & \\
      \hline
      CircleCIに残るソースコードのテスト実行結果 & \\
      \hline
      作業の見積もり作業時間と実際に作業した時間の比較 & \\
      \hline
    \end{tabular}
  \end{center}
\end{table}
