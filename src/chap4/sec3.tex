\section{実践と制約を踏まえたソフトウェア開発実習の授業提案}

本説では前節までのことを踏まえ,授業の目的と目標,前提条件をまとめた上で取り扱い内容と授業の評価方法を提案する.

\subsection{目的}

本研究で提案する授業は,アジャイルに参加するための知識と技術を身につけることを目的とする.

\subsection{目標}

本研究で提案する授業は,アジャイルとはどのようなものかを知り, アジャイルソフトウェア開発の為に用意された環境を用いてソフトウェア開発が出来るようになることを目標とする.

\subsection{取り扱い内容
  \label{tech-content}
}

本研究で提案する授業で取り扱う内容は以下のとおりである.

\begin{enumerate}
  \item アジャイルソフトウェア開発について
  \item バージョン管理システムを用いたソフトウェアのバージョン管理
  \item テスティングフレームワークを用いたテスト
  \item アプリケーションフレームワークを用いたソフトウェア開発
  \item 継続的インテグレーションシステムを利用した継続的改善
\end{enumerate}

\subsection{対象 \label{target}}

本研究で提案する授業は,情報系学部に就学する大学3年生程度の技術力を習得している学生,
具体的には以下ような技術を身につけている者を対象とする.

\begin{itemize}
  \item[・] 任意のプログラミング言語を用いて,オブジェクト指向のためのプログラミング方法を知っていること
  \item[・] コマンドラインからLinuxを扱うことに抵抗が無いこと
\end{itemize}

\subsection{実習について}

提案する授業の内、数回を実習の時間とし,授業時間では以下のことを毎週行う.

\begin{enumerate}
  \item 前週での改善のフィードバック
  \item ソフトウェアの問題点の列挙
  \item 各問題点の改善に必要な時間の予想
  \item 各問題点の優先順位づけ
  \item 改善する問題点の決定
  \item 質疑応答
\end{enumerate}

また授業時間外では受講者に問題点に対する改善を以下のフローを繰り返し行ってもらう.

\begin{enumerate}
  \item 問題点の確認
  \item 問題点に対するテストコードの作成
  \item 問題点の改善のためのコード記述
  \item 改善できているかの確認
  \item 継続的インテグレーションシステムでの確認
  \item 本番環境へのコード反映
\end{enumerate}

\subsection{評価方法}

受講者の評価は,実習に必要な技術の講義が終了後に行うソフトウェア改善実習の際に収集できる作業ログで行う.
収集できる作業ログとその評価方法については表\ref{tab:作業ログと評価方法}にまとめた.

\begin{table}[ht]
  \begin{center}
    \caption{作業ログと評価方法}
    \begin{tabular}{|c|c|}
      \hline
      収集できるログ & 評価方法 \\
      \hline
      バージョン管理システムを利用したソースコードの変更ログ & 改善事項に対して必要な行数で実現できているか \\
      \hline
      開発環境上で利用したコマンドの履歴 & 改善作業に対して正しくコマンドを叩けているかどうか \\
      \hline
      継続的インテグレーションシステムに残るテスト実行結果 & テストが落ちていないかどうか \\
      \hline
      見積もった作業時間 & \\
      \cline{1-1}
      実際に作業した時間 & 見積もりと実働の差 \\
      \hline
    \end{tabular}
    \label{tab:作業ログと評価方法}
  \end{center}
\end{table}
