\section{実践と制約を踏まえたソフトウェア開発実習の提案}

本説では前節までのことを踏まえ,授業の目的と目標,前提条件をまとめた上で取り扱い内容とシラバスと授業の評価方法を提案する.

\subsection{時間}

本研究で提案する授業は,大学設置基準\cite{univ_criteria}で制定されている単位,2単位分の授業とする.

\subsection{目的}

本研究で提案する授業は,アジャイルに参加するための知識と技術を身につけることを目標とする.

\subsection{目標}

本研究で提案する授業は,アジャイルとはどのようなものかを知り, Ruby on Railsを用いてソフトウェア開発が出来るようになることを目標とする.

\subsection{取り扱い内容
  \label{tech-content}
}

本研究で提案する授業で取り扱う内容は以下のとおりである.

\begin{enumerate}
  \item アジャイルソフトウェア開発について
  \item プログラミング言語 Ruby
  \item Gitを用いたソフトウェアのバージョン管理
  \item Rspecを用いたテスト
  \item Ruby On Rails を用いたソフトウェア開発
  \item 継続的インテグレーションシステムを利用した継続的改善
\end{enumerate}

上記を扱う上で選定する技術的なツールは表\ref{tab:技術選定}のとおりとなる.

\begin{table}[ht]
  \begin{center}
    \caption{技術選定}
    \begin{tabular}{|c|c|c|}
      \hline
      分類 & ツール名 & 備考 \\
      \hline
      プログラミング言語 & Ruby & \\
      \hline
      アプリケーションフレームワーク & Ruby on Rails & \\
      \hline
      テスティングフレームワーク & Rspec & \\
      \hline
      アプリケーションサーバー &  & \\
      \cline{1-1}\cline{3-3}
      Webサーバー & Heroku & \\
      \hline
      RDBMS & PostgreSQL & \\
      \hline
      バージョン管理システム & Git & \\
      \hline
      ソースコードホスティングサーバー & GitHub & \\
      \hline
      継続的インテグレーションシステム & CircleCI & \\
      \hline
    \end{tabular}
    \label{tab:技術選定}
  \end{center}
\end{table}

\subsection{対象}

本研究で提案する授業は,情報系学部に就学する大学3年生程度の技術力を習得している学生を対象にする.
具体的には以下のとおりである.

\begin{itemize}
  \item[・] HTMLやCSSの扱い方を知っていること
  \item[・] HTTPメソッドやセッション,クッキーを理解していること
  \item[・] 任意のプログラミング言語を用いて,オブジェクト指向のためのプログラミング方法を知っていること
  \item[・] コマンドラインからLinuxを扱うことに抵抗が無いこと
\end{itemize}

\subsection{計画}

本研究で提案する授業の計画は表\ref{tab:シラバス}のとおりとする.

\begin{table}[H]
%  \begin{center}
    \caption{シラバス}
    \begin{tabular}{|c|c|c|c|}
      \hline
      授業回数 & 内容 & \ref{tech-content}取り扱い内容との対応 \\%& 備考 \\
      \hline
      第1回 & ソフトウェア開発について & 1 \\%&  \\
      \hline
      第2回 & コマンドラインの利用方法 &  \\%&  開発環境の説明も含めた復習 \\
      \hline
      第3回 & Ruby基礎文法 &  2 \\%&  \\
      \hline
      第4回 & オブジェクト指向Ruby & 2 \\%& Rubyの説明も兼ねたオブジェクト指向の復習 \\
      \hline
      第5回 & Gitの使いかた & 3 \\%& \\
      \hline
      第6回 & GitHubとRspec & 2,3,4 \\%&  \\
      \hline
      第7回 & Ruby On Rails の始め方 & 5 \\%&  \\
      \hline
      第8回 & Ruby On Rails を利用したソフトウェアの作成 & 5 \\%& \\
      \hline
      第9回 & Ruby On Rails を利用したソフトウェアの改善 1 & 5 \\%& \\
      \hline
      第10回 & Ruby On Rails を利用したソフトウェアの改善 2 & 5 \\%& \\
      \hline
      第11回 & Ruby On Rails とテスト,継続的インテグレーション & 4,5,6 \\%& \\
      \hline
      第12回 & Ruby On Rails とテスト,継続的インテグレーション & 4,5,6 \\%& \\
      \hline
      第13回 & Ruby On Rails で作成されたソフトウェアの改善実習 & 全内容 \\%& 画像投稿,お気に入りシステムの改善\\
      \hline
      第14回 & Ruby On Rails で作成されたソフトウェアの改善実習 & 全内容 \\%& 画像投稿,お気に入りシステムの改善\\
      \hline
      第15回 & Ruby On Rails で作成されたソフトウェアの改善実習 & 全内容 \\%& 画像投稿,お気に入りシステムの改善\\
      \hline
    \end{tabular}
    \label{tab:シラバス}
%  \end{center}
\end{table}

\subsection{実習について}

提案する授業の内,第13〜15回の期間を実習の時間と,授業時間では以下のことを毎週行う.

\begin{enumerate}
\item ソフトウェアの問題点の列挙
\item 各問題点の改善に必要な時間の予想
\item 各問題点の優先順位づけ
\item 改善する問題点の決定
\end{enumerate}

また授業時間外では受講者に問題点に対する改善を以下のフローを繰り返し行ってもらう.

\begin{enumerate}
\item 問題点の確認
\item 問題点に対するテストコードの作成
\item 問題点の改善のためのコード記述
\item 改善できているかの確認
\item 継続的インテグレーションシステムでの確認
\item 本番環境へのコード反映
\end{enumerate}

\subsection{評価方法}

本研究で提案する授業効果の評価は,終盤の3回分の実習で行うソフトウェア改善の際に収集できる作業ログで行う.
収集できる作業ログとその評価方法については表\ref{tab:作業ログと評価方法}にまとめた.

\begin{table}[ht]
  \begin{center}
    \caption{作業ログと評価方法}
    \begin{tabular}{|c|c|}
      \hline
      収集できるログ & 評価方法 \\
      \hline
      Gitを利用したソースコードの変更ログ & 改善事項に対して必要な行数で実現できているか \\
      \hline
      Linux上で利用したコマンドの履歴 & 改善作業に対して正しくコマンドを叩けているかどうか \\
      \hline
      CircleCIに残るテスト実行結果 & テストが落ちていないかどうか \\
      \hline
      見積もった作業時間 & \\
      \cline{1-1}
      実際に作業した時間 & 見積もりと実働の差 \\
      \hline
    \end{tabular}
    \label{tab:作業ログと評価方法}
  \end{center}
\end{table}

上記の表の各評価方法について詳しく述べる.

\subsubsection{改善事項に対して必要な行数で実現できているか}

後で書く

\subsubsection{改善作業に対して正しくコマンドを叩けているかどうか}

後で書く

\subsubsection{テストが落ちていないかどうか}

後で書く

\subsubsection{見積もりと実働の差}

後で書く
