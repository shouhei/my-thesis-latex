\section{企業がアジャイル開発手法を採用する理由}

ディーン・レフィングウェルによる文献 \cite{leffingwell2010アジャイル開発の本質とスケールアップ}では,ウォーターフォールを行う際に必要な仮定が誤っているためにソフトウェア開発に失敗し,それに対してアジャイルではどのように対処するかを述べている.
以下はその文献で挙げられているウォーターフォール開発にあたっての仮定である。

(以下は抜粋だが,わかりにくいので補足が必要なので抜粋,引用を止めるかも)

\begin{quote}
  \begin{enumerate}
    \item 適切に定義された要求が既に存在し,時間さえ掛ければそれらを理解することが出来る
    \item 変更は小さく,管理できる
    \item システムの統合はうまくいく
    \item スケジュール通りに提供できる
  \end{enumerate}
\end{quote}

しかし,上記の仮定を満たすようなソフトウェア開発が行われることは少なく,結果としてウォーターフォールによるソフトウェア開発は失敗になると述べている.上記の誤った仮定に対応する為に,アジャイルでは以下のような仮定を設け,ウォーターフォールの欠点を克服している.

(以下は抜粋だが,わかりにくいので補足が必要なので抜粋,引用を止めるかも)

\begin{quote}
  \begin{enumerate}
    \item 私たちもしくは顧客が全ての要求を完全に理解できる,あるいは,前もって理解できるとは仮定しない
    \item 変更が小さく対処可能であると仮定しない.むしろ,常に変更があると仮定する.
    \item システムの統合は重要なプロセスで,リスクを減らすために必要であると仮定する.
    \item 一定の機能と一定の予定に基づいて,新しく,最新技術で,実績がなく,革新的で,リスクが多いソフトウェア開発プロジェクトを行うことが出来ると仮定しない.逆に,そうすることは本当に不可能であると仮定する.その代わりに,最も重要なフィーチャーはむしろ顧客の予想よりも早く顧客に提供できると仮定する.
  \end{enumerate}
\end{quote}

(だからアジャイルを導入してるんですよーって話をココに書く)
