\section{企業がアジャイル開発手法を採用する理由}

ディーン・レフィングウェルによる文献 \cite{leffingwell2010アジャイル開発の本質とスケールアップ}では,ウォーターフォールを行う際に必要な仮定が誤っているために開発に失敗し,失敗する理由に対してアジャイルではどのように対処するかを述べている.
その文献ではウォーターフォールを行う際の仮定として以下を挙げている.

(以下は抜粋だが,わかりにくいので補足が必要なので抜粋,引用を止めるかも)

\begin{quote}
  1. 適切に定義された要求が既に存在し,時間さえ掛ければそれらを理解することが出来る

  2. 変更は小さく,管理できる

  3. システムの統合はうまくいく

  4. スケジュール通りに提供できる
\end{quote}

しかし,上記の仮定を満たすようなソフトウェア開発が行われることは少なく,結果としてウォーターフォールは失敗となると述べている.誤った仮定に対応する為に,アジャイルは以下のような仮定を設け,ウォーターフォールの欠点を克服している.

(以下は抜粋だが,わかりにくいので補足が必要なので抜粋,引用を止めるかも)

\begin{quote}
  1. 私たちもしくは顧客が全ての要求を完全に理解できる,あるいは,前もって理解できるとは仮定しない

  2. 変更が小さく対処可能であると仮定しない.むしろ,常に変更があると仮定する.そして,変更に対応できるように少しずつ提供する.

  3. システムの統合は重要なプロセスで,リスクを減らすために必要であると仮定する.したがって,最初から統合し,そして継続的に統合を行う.「システムは,常に動作する」という方針に従い,デモンストレーションできる出荷可能な製品が常にあることを保証するように務める.

  4. 一定の機能と一定の予定に基づいて,新しく,最新技術で,実績がなく,革新的で,リスクが多いソフトウェア開発プロジェクトを行うことが出来ると仮定しない.逆に,そうすることは本当に不可能であると仮定する.その代わりに,最も重要なフィーチャーはむしろ顧客の予想よりも早く顧客に提供できると仮定する.そうすれば,ソリューションが正しいかどうか,すぐにフィードバックが得ることが出来る.顧客の迅速かつ直接のフィードバックからソリューションが正しくないことがわかったとしても,全てが失われるわけではない.この時点では非常に小さな投資がなされただけなので,継続的に提供を続けながら,ソリューションをリファクタリングし,素早く発展させていくことができる.余計な手戻りなしにこれらができる.

\end{quote}

つまり,企業は開発を成功させるための仮定が誤っているウォーターフォールを止め,アジャイルを導入している.
