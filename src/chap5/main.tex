\chapter{評価実験と結果}

本研究では前章\ref{proposal}で提案した授業について以下の点を明らかにしたい.

\begin{itemize}
  \item[・] 授業進行に無理がなかったか
  \item[・] 授業難易度が適切であったか
  \item[・] 受講者が授業内容を正しく理解しているのか
\end{itemize}

上記の点を明らかにするために,本研究では以下のアンケートと実験を行った.

\begin{itemize}
  \item[・] 実験実施前アンケート
  \item[・] 提案授業の実施,評価実験
  \item[・] 実験実施後アンケート
\end{itemize}

\section{受講前の受講者アンケート}

提案授業の受講前アンケートは,受講者の日頃の技術的な取り組みや,過去の経験,習得済み技能についてを明らかにするために実施した.
本評価実験の被験者のおおよその属性としては,(後で書く)である.
アンケートの詳細結果はAppendixに掲載する.

\section{提案授業の実施,評価}

前章\ref{proposal}で提案した授業を実施し,各種作業ログと改善後のソフトウェアのソースコードを収集し,各受講者にたいして,前章で述べた評価基準を用いて評価した.
結果として,(後で書く)となった.

\section{受講後の受講者アンケート}

(後で書く)
