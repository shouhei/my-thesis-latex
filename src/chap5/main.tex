\chapter{評価実験と結果,考察}

本研究では前章\ref{proposal}で提案した授業について以下の点を明らかにしたい.

\begin{itemize}
  \item[・] 授業進行に無理がなかったか
  \item[・] 授業難易度が適切であったか
  \item[・] 受講者が授業内容を正しく理解しているのか
\end{itemize}

\section{実験とアンケート}

先ほど挙げた点を明らかにするために,本研究では以下のアンケートと実験を行った.

\begin{itemize}
  \item[・] 実験実施前アンケート
  \item[・] 提案授業の実施,評価実験
  \item[・] 実験実施後アンケート
\end{itemize}

\subsection{受講前の受講者アンケート}

提案授業の受講前アンケートは,受講者の日頃の技術的な取り組みや,過去の経験,習得済み技能についてを明らかにするために実施した.
本評価実験の被験者のおおよその属性としては,(後で書く)である.
アンケートの詳細結果はAppendixに掲載する.

\subsection{提案授業の実施,評価}

前章\ref{proposal}で提案した授業を実施し,各種作業ログと改善後のソフトウェアのソースコードを収集し,各受講者にたいして,前章で述べた評価基準を用いて評価した.
結果として,(後で書く)となった.

\subsection{受講後の受講者アンケート}

(後で書く)

\section{考察}

本節では前節での評価実験での結果を受け提案する授業の継続すべき点,問題点,改良点について述べる.

\subsection{問題点}

とりあえず箇条書き

\begin{itemize}
  \item[・] 技術選定の余地があったのではないか?
  \item[・] 1コマとしては取り扱い分量が多い
    \begin{itemize}
    \item[・] 以下本来取り扱うべきmin
      \begin{itemize}
        \item[・] WAFの使いかた
        \item[・] WAFを利用した際のテスト
        \item[・] 継続的インテグレーションの利用
      \end{itemize}
    \item[・] 別授業にするべき
      \begin{itemize}
        \item[・] プログラミング言語の扱い方
        \item[・] コマンドラインの利用方法
        \item[・] バージョン管理の方法
      \end{itemize}
    \end{itemize}
  \item[・] 受講人数が増えたら対応できない
    \begin{itemize}
      \item[・] TAの導入
      \item[・] 産学連携を視野に入れる
    \end{itemize}
\end{itemize}

\subsection{改良可能な点}

とりあえず箇条書き

\begin{itemize}
\item[・] 利用する*aaSの選定
  \begin{itemize}
    \item[・] GitHub Education (プライベートリポジトリが使えた)
    \item[・] GitHub Student Developer Pack (Travice CIがプライベートリポジトリで使えた)
  \end{itemize}
\item[・] 改善するソフトウェアのお題
\end{itemize}
